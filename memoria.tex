\documentclass[10pt,a4paper,spanish]{article}

\usepackage[spanish]{babel}
\usepackage[utf8]{inputenc}
\usepackage{amsmath, amsthm}
\usepackage{amsfonts, amssymb, latexsym}
\usepackage{enumerate}
% \usepackage[official]{eurosym}
\usepackage{graphicx}
\usepackage[usenames, dvipsnames]{color}
\usepackage{colortbl}
\usepackage{multirow}
\usepackage{fancyhdr}
\usepackage[all]{xy}
% \usepackage{minted}
\usepackage{float}
\usepackage{subfigure}
\usepackage{tikz}
\usepackage{pgfplots}
\usepackage{cancel}
\pgfplotsset{compat=1.5}

\usepackage[top=2.5cm, bottom=2.5cm, left=3cm, right=3cm]{geometry}

\usepackage[bookmarks=true,
            bookmarksnumbered=false, % true means bookmarks in
                                     % left window are numbered
            bookmarksopen=false,     % true means only level 1
                                     % are displayed.
            colorlinks=true,
            linkcolor=red,
            citecolor=blue]{hyperref}

\newcommand{\HRule}{\rule{\linewidth}{0.5mm}} % regla horizontal para  el titulo

\pagestyle{plain}
%con esto nos aseguramos de que las cabeceras de capítulo y de sección vayan en minúsculas

\fancyhf{} %borra cabecera y pie actuales
% \fancyhead[LE,RO]{}
% \fancyhead[LO]{}
\fancyfoot[C]{\thepage}
% \renewcommand{\headrulewidth}{0.5pt}
% \renewcommand{\footrulewidth}{0pt}
% \addtolength{\headheight}{0.5pt} %espacio para la raya
% \fancypagestyle{plain}{%
%       \fancyhead{} %elimina cabeceras en páginas "plain"
%       \renewcommand{\headrulewidth}{0pt} %así como la raya
% }

% %%%%% Para cambiar el tipo de letra en el título de la sección %%%%%%%%%%%
% \usepackage{sectsty}
% \chapterfont{\fontfamily{frc}\selectfont}
% \sectionfont{\fontfamily{pag}\selectfont}
% \subsectionfont{\fontfamily{pag}\selectfont}
% \subsubsectionfont{\fontfamily{pag}\selectfont}

% \newmintedfile[mycplusplus]{c++}{
%     linenos,
%     numbersep=5pt,
%     gobble=0,
%     frame=lines,
%     framesep=2mm,
%     tabsize=3,
% }

% \newmintedfile[mypython]{python}{
%     linenos,
%     numbersep=5pt,
%     gobble=0,
%     frame=lines,
%     framesep=2mm,
%     tabsize=3,
% }

\definecolor{amaranth}{rgb}{0.9, 0.17, 0.31}

\usepackage{arev}
\usepackage[T1]{fontenc}

\setlength{\parindent}{0pt}
\setlength{\parskip}{1ex plus 0.5ex minus 0.2ex}

% \usepackage{titlesec}

% % \titleformat{\chapter}{\normalfont\huge\center}{--- \thechapter ---}{20pt}{}

% \titleformat
% {\chapter} % command
% [display] % shape
% {\Huge\center\bfseries} % format
% {--- \thechapter ---} % label
% {0.5ex} % sep
% {
%     \rule{\textwidth}{1pt}
%     \vspace{1ex}
%     \centering
% } % before-code
% [
% \vspace{-0.5ex}%
% \rule{\textwidth}{0.3pt}
% ] % after-code

%Definimos autor y título
\title{\Huge Seguridad y Uso de SSH}

\begin{document}

\begin{center}
{\Huge Seguridad y uso de SSH}
\subsubsection*{Resumen}
\end{center}

\section{¿Qué es SSH?}
SHH es un protocolo hecho para comunicar dos ordenadores de forma segura. Esto quiere decir que SSH, antes de enviar datos, los encripta y cuando llegan a la máquina destino los desencripta. Así, conseguimos una encriptación transparente para el usuario pero a la vez una comunicación segura entre dos máquinas. SSH funciona con una arquitectura cliente-servidor.

Además de la encriptación, SSH también se ocupa de verficar la identidad de la persona que se quiere conectar y de garantizar que los datos que se envían llegan correctamente. Si una tercera persona captura los datos y los modifica, SSH lo detecta (\cite{sshbiblio}).

\section{Comparación con Telnet}
Tal y como se indica en \cite{sshbiblio}, Telnet nos permite también conectarnos en un ordenador de forma remota, pero tiene un defecto: transmite nuestro nombre de usuario y contraseña en texto plano a través de Internet donde una tercera persona puede interceptarlo. Además, toda la sesión Telnet realizada puede ser leída por cualquiera que la intercepte pues también se transmite en texto plano.

Para hacer un ejemplo práctico, hemos creado dos máquinas virtuales que vamos a comunicar a través del servicio Telnet y SSH. Para comprobar cómo podemos extraer los datos de inicio de sesión, vamos a realizar una escucha usando \textit{Wireshark}, y codificando los paquetes como Telnet. Como se puede ver en la \hyperref[telnet1]{Figura \ref*{telnet1}}, hemos capturado todos los paquetes que se han generado durante el inicio de sesión a través de Telnet. 

Una vez que se tiene esto, entrando en \verb|Analyze > Follow TCP Stream| podemos ver el flujo de caracteres que ha habido durante la comunicación que ha seguido el protocolo de red TCP. Como se ve en \hyperref[telnet2]{Figura \ref*{telnet2}}, en este flujo podemos ver claramente el usuario y contraseña que hemos usado para hacer $login$.

\begin{figure}[H]
    \centering
    
    \mbox {
        \subfigure[Paquetes capturados para la conexión Telnet.]{
        \includegraphics[width=0.5\textwidth]{telnet1}
        \label{telnet1}
    }
    \qquad
    
    \subfigure[TCP Stream: en el se ve el user y password de la conexión.] {
        \includegraphics[width=0.51\textwidth]{telnet2}
        \label{telnet2}
        }
    }
    \caption{Escucha realizada para la conexión Telnet.}
    \label{telnet}
\end{figure}

\bibliography{memoria} 
\bibliographystyle{siam}

\end{document}